\begin{document}

%Plan {{{ 
% Title
Make sure that I only talk about areas I am really interested in and want to talk about. 


State of the art experimental apparatus for fast entangling gates in trapped multi-ion crystals

11 Page limit

(Short) intro to trapped ion QC and Theory on Fast Gates schemes: ~ 2 pages

    * Motivating why QC important. ~ 1 paragraph

    * Trapped Ion QC - General idea of spin (ion) coupled with HO (Trapping potential) and how this satisfies QC requirements. ~ 1-2 paragraph

    * Entangling gates MS gate. Statement of Hamiltonian and how this is experimentally realised. ~ 1 paragraph

    * Fast Gate Schemes (Non-Adiabatic Entanglement). Amplitude shaped pulses. ~ 1 sentence and reference
        Why we want to move to quadrupole optical transition rather than Raman (Scattering error and squeezing term)

    * Carrier Nulling. (excerpt from paper starts) ~ 2 paragraphs

 

Experimental results from Carrier Nulling (excerpts from paper): ~ 2+ pages

    * Description of Blade ~ 1 figure + 1 paragraph

    * Phase control ~ 1 paragraph

    * Proof of principle on Blade apparatus ~ 2 paragraph (statement of results) 2 figures

    Could dependence on AC shift with position/field intensity be an issue? ~ 1 paragraph

    RBM work ~ few sentences

 

FastGates buildup: ~ 4 pages

    * Motivation for why we want a new system (compare with Blade) ~ 1 paragraph

    * Description of FastGates – central figure of full physical experiment and figure of control systems ~ 2 paragraphs large figure
        In effect list all the benefits over Blade - and how it fits into requirements for NISQ?

    Now looking at components of new system from center outward:

    * Ca40 energy diagram  ~ 1 paragraph and energy level figure
        Simple energy structure. Less spectator modes compared to Ca43 -> less off resonant transitions. But sensitive to B field.

    * NPL trap, trap frequencies, substrate bias for crystal rotation ~ 1 paragraph 1 figure
        What trap freq do we want for fast gates 

    * In vacuum system details ~ 1 paragraph 1 figure

    * Dual Optical Access High NA system ~ few sentences

    * Single Ion Addressing with AOD, power requirements ~ 1 paragraph
        This is potentially a drain of time – time bound looking into this.

    * Addressing and Readout optics design? ~ short 1 figure

    * Extension to standing wave single ion addressing – ideas/design for phase feedback ~ few sentences 1 figure?

    * 729 system design, PDH locking, FNC ~ 1 paragraph

   

Outlook: ~ 1 page

    * Overall goals of project
        - Optical phase control of laser field at the ion.
        - Fastgates on multi ion chain.

    * Immediate tasks: Finishing vacuum work, trap on table. Trapping ions 

    * Proposed first experiments? 
        1st paper: EOM to keep stark shift the same as pulse amplitude changes 
        Explore this for 1 day
            Ask about scheme 
            This is a problem with gates with multiple pulses. 
                Theory on this

What will my next 6 months, 1 yr, 2 yrs look like? 
%}}}

%Abstract {{{
Abstract
%}}}

%Introduction {{{
Introduction
Previously the physicist was limited to thought experiments on the
nature of highly coherent quantum effects. It was even thought to be
inconceivable for single atoms to be probed and entangled. However the
advent of Ion traps, specifically Paul and Penning traps, have enabled
the experimental exploration of Atomic and Laser physics and some of
the most precise measurements of physical constants.  Decoherence may
be seen as measurement of a state by the environment. Ion traps allow
the decoupling of single ions to the surrounding environment and
thus...
With the high degree of experimental control Ion trap systems coupled
with low linewidth lasers provides, trapped ions are a popular
platform for enabling Quantum Computation.

    *Trapped Ion QC
    Paul trap details
    General idea of spin coupled with HO
    *Entangling gates MS gate, light shift gate
    *Non-Adiabatic interactions
    *Fast Gate Schemes

%}}}

%Experimental Details {{{
Experimental Details


\subtitle{Ion Trapping Apparatus}


The technical complexity of ion trapping experiments may be reduced to
solving two problems: Controlling the state of the ion (both internal
and motional); and controlling the environment the ion is within.
Our Ion trap experiments consist of: an atomic source, a trap, a
vacuum system encasing these, external magnetic field coils, and
lasers for ionization, cooling, repumping, stateprep, coherent
control and readout.
As with all ventures in experimental physics, as technologies mature, so
too do the capabilities and scope of our apparatus.  Here we shall
describe the ``Old'' apparatus, known in short as ``Blade'', where
proof of principle fast gate schemes have been tested. The limitations
of ``Blade'' for further fast gate work will be made apparant and
the new proposed system, known as ``FastGates'', will be described. 

\subtitle{Ion and Trap}

As mentioned, the overall system we desire is a spin coupled to a
spring. Our spin in this case being a Hydrogen-like ion, and the
spring being the harmonic motion of the ions within the trapping
potential. The ion traps we use to creeate such a potential are linear
Paul traps, a schematic of such is shown in FigureX. As explained by
Earnshaw's theorem, a stable stationary point in 3D can not be
realized using static electric field. Therefore a Paul trap utilizes
an oscillating electric field to create a trapping
pseudopotential. There are various geometries for realizing a paul
trap, shown in Figure X are: A macro 3D Blade trap; a surface trap;
and a microfabricated multilayer trap.  A Blade trap, as is used in
the ``Blade'' apparatus, has axial confinement created by DC end caps
and radial confinement by supplying an oscillating RF on the blades.
In ``Blade'' the ion endcap distance is $1.15$~mm, and ion-blade
distance is $0.5$~mm. Typical operating frequency for the RF
electrodes of the ``Blade'' trap are $28.0133$ MHz leading to an axial
ion frequency of $1.860$ MHz and radial frequencies of $4.077$ MHz and
$4.341$ MHz.

For the sake of comparison, recently the surface style linear Paul
trap has gained popularity due to the maturity of chip fabrication
technologies and the potential route to scalability this offers. In
the surface trap, the 3D blade and endcap geometry of the ``macro''
trap is effectively projected onto a 2D surface. The stable point of
such a trap is typically on the order of $50$ um from the chip
surface. The ease of fabrication of surface traps has allowed the
creation of complicated multizone devices with many DC electrodes.
These multizone traps enable the shuttling of ions, a requirement for
Quantum CCD type architectures. However these benefits come at the
cost of trapping potential. Heating of an ion is XXproportionalXX to
the ion electrode distance, however so too is the trapping
potential. This leads to a compromise of distance... surface trap
creates a poor approx of harmonic potential... Therefore weak trap and
high heating rates compared to a macro 3D blade trap. Heating rates of
HOA2: axial and radial frequencies...

A microfab 3D trap [See et al], as will be used in the ``FastGates''
apparatus, brings together the advantages of chip fabrication as well
as the low heating rates and high trapping fields of a 3D style
trap. This is achieved by a multilayer chip as shown in figureX. The
radial trapping is provided by RF rails on opposite diagonals of the
slit whilst axial trapping may be realized by DC electrodes on both
surfaces. The Ion electrode distance is now of the order $200$ um,
meaning lower heating, whilst the more optimal 3D geometry allows for
a deep potential at this distance. The microfabrication techniques
also allow a segmented design suitable for multizone operations and
ion shuttling. Heating rates... and calculate axial and radial modes!
Further, the ion being located within this slit allows for dual high
NA optical access (NA = XX), which is an important factor for our
proposed single addressing standing wave experiment.

\subtitle{Laser systems}
We have described the trapping of an ion, now we must look at our strategies for manipulating the ions internal states and the collective motion of ion strings. Our key tool for this is the use of lasers...
Figure X shows the energy level structure of Ca40+ and the transition frequencies we require for our experiment.

393 and 432 - PI 
397 - Cooling, readout
729 Quadrupole - Coherent control
854 and 866 - Repumping

\subtitle{The vacuum system}

Here we shall describe the instrumentation required, and being constructed, for decouping the ion from any unwanted external environments. Our primary tools for this are working under Ultra High Vacuum (UHV), and using electro magenetic shielding. The UHV system consists of a main experimental chamber comprised of a CF100 Octagon and pumps





%% Assuming our trapping potential to be quadratic, we have a spin system
%% coupled to a spring. The Jaynes Cummings Hamiltonian,
%% $$ H = H_{spin} + H_{HO} + H_{Int}, $$
%% summarises this coupled system.




Ion source,
Trap,
Laser addressing

second problem

Our Ion trapping experiments require control over the ion, the trap
Require:
Ion, Trap
Control over:
Internal state of ion, external environment of the ion

In a reductionist viewpoint, 
Description of Blade and limitations Proof of principle experiments
have been completed on Blade, a blade-style ion trap. Blade has a few
limitations for the exploration of fast gates by the above described
Cnulled method, motivating the design of a new system.  Blade has dual
optical access of the ions however, only as 45 degrees to the chain
axial direction. Only global laser addressing of the chain is possible
limiting the intensity seen at the ion. The simpler blade style trap,
has limited control over creation of electric potentials meaning
shuttling of ions is not practical.


\subtitle{FastGates Apparatus}

Here we describe the design of the new ``FastGates'' system which is
tailored for the exploration of fast, non-adiabatic entangling
gates. Figure X shows a schematic of the vacuum can of ``FastGates''
with the addressing directions and magnetic field highlighted. Ca40
was chosen for initial experiments due to its simple energy level
structure, figure X, without hyperfine levels and with the option for
a quadrupole qubit between the S and D levels. An external magnetic
field of 5G is applied to define our Zeeman sublevels, this low field
will not allow state selective addressing by frequency, however allows
for polarization selective addressing. *** Check if 729 will actually
have linewidth for frequency addressing? *** The isotope having 0
nuclear spin and hence no hyperfine levels greatly simplifies control
schemes however precludes the option of using magnetically insensitive
``clock'' qubits. To ensure we do not greatly limit coherence time of
our quadrupole transition we use a MuMetal enclosure to suppress stray
environmental magnetic fields.


%}}}

%Results and Discussion {{{
Results and Discussion
    *Carrier nulling on Blade
%}}}

%Outlook {{{
Outlook
    Current state of building up apparatus
    Proposed first experiments?

%}}}
  
\end{document}
