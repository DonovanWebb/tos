\documentclass{article}
\begin{document}

%Plan {{{ 
% Title
Make sure that I only talk about areas I am really interested in and want to talk about. 


State of the art experimental apparatus for fast entangling gates in trapped multi-ion crystals

11 Page limit

(Short) intro to trapped ion QC and Theory on Fast Gates schemes: ~ 2 pages

    * Motivating why QC important. ~ 1 paragraph

    * Trapped Ion QC - General idea of spin (ion) coupled with HO (Trapping potential) and how this satisfies QC requirements. ~ 1-2 paragraph

    * Entangling gates MS gate. Statement of Hamiltonian and how this is experimentally realised. ~ 1 paragraph

    * Fast Gate Schemes (Non-Adiabatic Entanglement). Amplitude shaped pulses. ~ 1 sentence and reference
        Why we want to move to quadrupole optical transition rather than Raman (Scattering error and squeezing term)

    * Carrier Nulling. (excerpt from paper starts) ~ 2 paragraphs

 

Experimental results from Carrier Nulling (excerpts from paper): ~ 2+ pages

    * Description of Blade ~ 1 figure + 1 paragraph

    * Phase control ~ 1 paragraph

    * Proof of principle on Blade apparatus ~ 2 paragraph (statement of results) 2 figures

    Could dependence on AC shift with position/field intensity be an issue? ~ 1 paragraph

    RBM work ~ few sentences

 

FastGates buildup: ~ 4 pages

    * Motivation for why we want a new system (compare with Blade) ~ 1 paragraph

    * Description of FastGates – central figure of full physical experiment and figure of control systems ~ 2 paragraphs large figure
        In effect list all the benefits over Blade - and how it fits into requirements for NISQ?

    Now looking at components of new system from center outward:

    * Ca40 energy diagram  ~ 1 paragraph and energy level figure
        Simple energy structure. Less spectator modes compared to Ca43 -> less off resonant transitions. But sensitive to B field.

    * NPL trap, trap frequencies, substrate bias for crystal rotation ~ 1 paragraph 1 figure
        What trap freq do we want for fast gates 

    * In vacuum system details ~ 1 paragraph 1 figure

    * Dual Optical Access High NA system ~ few sentences

    * Single Ion Addressing with AOD, power requirements ~ 1 paragraph
        This is potentially a drain of time – time bound looking into this.

    * Addressing and Readout optics design? ~ short 1 figure

    * Extension to standing wave single ion addressing – ideas/design for phase feedback ~ few sentences 1 figure?

    * 729 system design, PDH locking, FNC ~ 1 paragraph

   

Outlook: ~ 1 page

    * Overall goals of project
        - Optical phase control of laser field at the ion.
        - Fastgates on multi ion chain.

    * Immediate tasks: Finishing vacuum work, trap on table. Trapping ions 

    * Proposed first experiments? 
        1st paper: EOM to keep stark shift the same as pulse amplitude changes 
        Explore this for 1 day
            Ask about scheme 
            This is a problem with gates with multiple pulses. 
                Theory on this

What will my next 6 months, 1 yr, 2 yrs look like? 
%}}}

%Abstract {{{
Abstract
%}}}

%Introduction {{{
Introduction
Previously the physicist was limited to thought experiments on the
nature of highly coherent quantum effects. It was even thought to be
inconceivable for single atoms to be probed and entangled. However the
advent of Ion traps, specifically Paul and Penning traps, have enabled
the experimental exploration of Atomic and Laser physics and some of
the most precise measurements of physical constants.  Decoherence may
be seen as measurement of a state by the environment. Ion traps allow
the decoupling of single ions to the surrounding environment and
thus...
With the high degree of experimental control Ion trap systems coupled
with low linewidth lasers provides, trapped ions are a popular
platform for enabling Quantum Computation.

    *Trapped Ion QC
    Paul trap details
    General idea of spin coupled with HO
    *Entangling gates MS gate, light shift gate
    *Non-Adiabatic interactions
    *Fast Gate Schemes

%}}}

%Experimental Details {{{
Experimental Details


subtitle Ion Trapping Apparatus\\

The technical complexity of ion trapping experiments may be reduced to
solving two problems: Controlling the state of the ion (both internal
and motional); and controlling the environment the ion is within.
Our Ion trap experiments consist of: an atomic source, a trap, a
vacuum system encasing these, external magnetic field coils, and
lasers for ionization, cooling, repumping, stateprep, coherent
control and readout.
As with all ventures in experimental physics, as technologies mature, so
too do the capabilities and scope of our apparatus.  Here we shall
describe the ``Old'' apparatus, known in short as ``Blade'', where
proof of principle fast gate schemes have been tested. The limitations
of ``Blade'' for further fast gate work will be made apparant and
the new proposed system, known as ``FastGates'', will be described. 

subtitle Ion and Trap\\

As mentioned, the overall system we desire is a spin coupled to a
spring. Our spin in this case being a Hydrogen-like ion, and the
spring being the harmonic motion of the ions within the trapping
potential. The ion traps we use to creeate such a potential are linear
Paul traps, a schematic of such is shown in FigureX. As explained by
Earnshaw's theorem, a stable stationary point in 3D can not be
realized using static electric field. Therefore a Paul trap utilizes
an oscillating electric field to create a trapping
pseudopotential. There are various geometries for realizing a paul
trap, shown in Figure X are: A macro 3D Blade trap; a surface trap;
and a microfabricated multilayer trap.  A Blade trap, as is used in
the ``Blade'' apparatus, has axial confinement created by DC end caps
and radial confinement by supplying an oscillating RF on the blades.
In ``Blade'' the ion endcap distance is $1.15$~mm, and ion-blade
distance is $0.5$~mm. Typical operating frequency for the RF
electrodes of the ``Blade'' trap are $28.0133$ MHz leading to an axial
ion frequency of $1.860$ MHz and radial frequencies of $4.077$ MHz and
$4.341$ MHz.

For the sake of comparison, recently the surface style linear Paul
trap has gained popularity due to the maturity of chip fabrication
technologies and the potential route to scalability this offers. In
the surface trap, the 3D blade and endcap geometry of the ``macro''
trap is effectively projected onto a 2D surface. The stable point of
such a trap is typically on the order of $50$ um from the chip
surface. The ease of fabrication of surface traps has allowed the
creation of complicated multizone devices with many DC electrodes.
These multizone traps enable the shuttling of ions, a requirement for
Quantum CCD type architectures. However these benefits come at the
cost of trapping potential. Heating of an ion is XXproportionalXX to
the ion electrode distance, however so too is the trapping
potential. This leads to a compromise of distance... surface trap
creates a poor approx of harmonic potential... Therefore weak trap and
high heating rates compared to a macro 3D blade trap. Heating rates of
HOA2: axial and radial frequencies...

A microfab 3D trap [See et al and Wilpers 2012], as will be used in
the ``FastGates'' apparatus, brings together the advantages of chip
fabrication as well as the low heating rates and high trapping fields
of a 3D style trap. This is achieved by a multilayer chip as shown in
figureX. The radial trapping is provided by RF rails on opposite
diagonals of the slit whilst axial trapping may be realized by DC
electrodes on both surfaces. The Ion electrode distance is now of the
order $200$ um, meaning lower heating, whilst the more optimal 3D
geometry allows for a deep potential at this distance. The
microfabrication techniques also allow a segmented design suitable for
multizone operations and ion shuttling.  The 3D confining potential
leads to motion of the ions following the Hamiltonian

$$ H = \sum_{i=1}^N \frac{m}{2}(w_x^2x_i^2 + w_y^2y_i^2 + w_z^2z_i^2 + \frac{|p_i|^2}{m^2}) + \sum_{i=1}^N\sum_{j>i}\frac{e^2}{4\pi\epsilon_0|r_i - r_j|}$$,

where $w_v$ are the mode frequencies in the three dimensional
coordinated. We define $z$ to be the axial direction of the trap and
typically assume $w_z$ << $w_x$, $w_y$ to allow a 1D ion crystal to
lie along the axial direction of the trap. We aim for an axial ion
separation of around $5$~um which, for $^+$Ca_{40} ions means a
trapping potential of $w_z \approx 2\pi * 1.6$~MHz. We plan for around
$5$~MHz for our radial frequencies as we will use one such mode for
implementing two-qubit entangling gates. This higher frequency is for
a few reasons: The doppler cooling limit ($\overline{n} = \Gamma/w$,
where $\Gamma$ is the transition linewidth and $w$ is the
frequency of the mode being cooled) goes with the reciprocal of the
mode frequency and so higher mode frequencies leads to a lower
temperature after cooling; a higher c.o.m. radial mode leads to better
separation of radial modes in a multiion crystal => simpler
implementation of fast gate schemes where multiple motional modes are
all excited; can push to faster gates?  Given a model of the NPL trap,
we require an $\Omega_{RF} = XX20$~MHz with a driving amplitude of
$180$~V to find a solution with axial frequency of 1.6MHz and a radial
frequency $w_x = 5$~MHz.
One foreseeable issue with this arrangement becomes apparant when
considering the Matthiu equation representing trapping with
pseudo-potential. There are areas of stability and instability which
can be quantified with the factor $q$. In Ion traps it has been shown
that areas of stability exist for $q<0.9$, however typical values of
$q$ for trapping and cooling ions are considerably smaller.
$q = 2*\sqrt{2}w_{ax}/\Omega_{RF}$.
For the proposed plan of $20$~MHz RF and $5$~MHz radial frequency we
would have $q = 0.7$ which although is still stable, may not be able
to practically trap from a hot source of ions. Therefore we will
likely trap at a lower RF amplitude, lowering the radial frequencies
to around $2$~MHz where $q<0.3$ and then ramp up to a tighter trap for
efficient doppler cooling and fast gate experiments.

Some simulations to find solutions with decent axial and radial modes and figures.

Heating rates... Further, the ion being located within this
slit allows for dual high NA optical access (NA = XX), which is an
important factor for our proposed single addressing standing wave
experiment.

subtitle Laser systems\\

We have described the trapping of an ion, now we must look at our
strategies for manipulating the internal states and collective motion
of ion strings. Our key tool for this is the use of lasers as we can
create highly localised, strong electric field amplitudes and
gradients. Coupling to spin: carrier interaction, Rabi flopping, pi
pulses. Coupling to motion: description of sidebands.
Figure X shows the energy level structure of Ca40+, note the plethora
of available transitions available to realize the control over our ion
strings. Here we describe what transitions we have chosen for what
task.
Figure X shows the entry points of the laser systems into the trap
chamber.

729 - Coherent control
We will use the optical S->D state with the 729 quadrupole transition
to define our qubit due to the D state being a metastable state:
i.e. the transition from D to S is dipole forbidden as they are the
same parity DelL != +-1. The lifetime of this state is ~X700 ms,
giving a greater upper bound of the qubit coherence time than if a
dipole transition is used. This long lived state means that it is a
narrow linewidth transition (Heisenberg uncertainty) and so we must
use a narrow linewidth laser to efficiently control.  High power
required as lamb dicke factor is low for 729 transition.  We use a
pumped Ti:Saph system coupled to a high finesse cavity to achieve
this.  Verdi 532 -> Solstis 729 PDH locking with Stable laser systems
lock box. Below a schematic of our 729 laser system is
shown. Strontium is used in Blade apparatus and has all analogous
transitions. Here our quadrupole transition is at 674 nm.  FNC to
cavity and to laser lab. Power stabilization on the feedback AOM. High
bandwidth PID feedback loop pushing the servo bump away from area of
experimental interest. Acoustic isolation using foam insulated box.
Trial of Verdi C unsuccesful so far not achieving stable lasing with
solstis.
Improvements over Blade:
- 729 more convenient Ti:Saph freq -> Higher power, low noise
- access parallel to radial mode, Blade is 45 deg -> High Lamb Dicke
factor
- 729 more convenient for putting in fibre -> low charging


393 and 432 - PI 
Two step photoionization for isotope selectivity.

397 - Doppler Cooling, Fluorescent readout
Dipole transition for Doppler cooling as want fast scattering. Also
want higher freq light for more momentum transfer.
Fluor read out so that |0> bright and |1> dark.

854 and 866 - Repumping as branching ratio from P to S and D. So
during cooling and readout we lose population to these levels.

All of the above systems are Toptica Diode lasers coupled to a cavity
for PDH locking.


subtitle The vacuum system\\

Here we shall describe the instrumentation required, and being
constructed, for decouping the ion from any unwanted external
environments. Our primary tools for this are working under Ultra High
Vacuum (UHV) < $10^{-11}$ mbar, and using electro magenetic
shielding. The UHV system consists of a main experimental chamber
comprised of a CF100 Octagon and pumps. For UHV, we use protocols
described in XXX and must use appropriate materials within the
chamber. Note, as mentioned the NPL trap is a microfab 3D trap, our
ion therefore is located between the two planes within a slit. the 729
system requires High optical access for single addressing (see below)
and this limits the available entry points of the remaining beams. We
therefore are using in vacuum prisms to bounce the light into the
slit. We must use vacuum compatible glues and coatings of the
prisms. Figure X shows an image in the cleanroom of the prisms in our
system.
Also in the system:
Oven with thermocouple.
Trap PCB with filter capacitors.
Cabling to both.
Interposer between trap and PCB, using fuzz buttons.
Now pump system:
attached ion pump,
Ti:Sublimation pump,
Ion Gauge
Valve to connect to external pump system

design considerations and 

MuMetal shielding to suppress external magnetic fields. Using XX3mm
thick two layer system with a quoted magnetic field reduction of 100.


subtitle Single Addressing\\

729 High NA (0.6) lens we can achieve waist radius of < 1 um.  With
our axial trap freq of XXX we get ion spacing of ~ 5um.  Using AOD
system we can traverse this ion chain.  Description of lens system
from AOD to ions.
Description on the mechanism on how AOD works (this is the same as an
AOM). Comparison between this and a fixed waveguide array. What
equations we need to find number of resolvable spots of the AOD
system. AOD have extra programmable control than waveguides which is
excellent as we are in an exploratory regime where we may want to
alter ion spacing/Dont want to use quartic potentials to make ions
evenly spaced.
We will use a crossed AOD design so that we have no overall frequency
shift as we scan along the ion chain.
Compact design to fit beam path within MuMetal box.
We want to create single addressing standing wave so must consider
stabilization technique from AOD. 
There is minimal path length differece compared to waveguide which is
ideal as we can stabilise at some central frequency and should be
stable at all ion locations.
Quick calculation to look at path length diff in wavlengths between
two extrema points on the ion chain. Note that this will give some
fixed relationship between the two beams if we assume that the small
seperation of beams paths is negligible (air density and current in
close proximity should be related).
Using two RF freq so that we can address two ions at the same
time. Initially looking at this we see that supplying two freqs and
amplifying we dont have crazy cross term amplitudes.
However addressing multiple ions at the same time comes at the cost of
photon freq cross terms i.e. spots that we dont want that are off
plane to our ions. This has two bad effects: Can get crosstalk to
other ions on the chain and lose power in our 729 system. First effect
we can mitigate by putting AOD at a > 45 deg angle (60 deg?) this
pushes the unwanted spots further from the chain. (sep to ion is
sqrt2/2 when at 45 deg). There is no easy way to mitigate the power
loss as two freq photons are barely distinguishable (maybe look again
at the two wavelength design aods). So this may limit us to only
addressing 2 or three ions at a time and using a global addressing
system through the prisms if all ions need to be addressed. Quick
power calculation though means we still have XXmW at each ion with an
intensity of XX which could drive CNulled gates at a speed of X.





%% Assuming our trapping potential to be quadratic, we have a spin system
%% coupled to a spring. The Jaynes Cummings Hamiltonian,
%% $$ H = H_{spin} + H_{HO} + H_{Int}, $$
%% summarises this coupled system.






In a reductionist viewpoint, 
Description of Blade and limitations Proof of principle experiments
have been completed on Blade, a blade-style ion trap. Blade has a few
limitations for the exploration of fast gates by the above described
Cnulled method, motivating the design of a new system.  Blade has dual
optical access of the ions however, only as 45 degrees to the chain
axial direction. Only global laser addressing of the chain is possible
limiting the intensity seen at the ion. The simpler blade style trap,
has limited control over creation of electric potentials meaning
shuttling of ions is not practical.


subtitle FastGates Apparatus\\

Here we describe the design of the new ``FastGates'' system which is
tailored for the exploration of fast, non-adiabatic entangling
gates. Figure X shows a schematic of the vacuum can of ``FastGates''
with the addressing directions and magnetic field highlighted. Ca40
was chosen for initial experiments due to its simple energy level
structure, figure X, without hyperfine levels and with the option for
a quadrupole qubit between the S and D levels. An external magnetic
field of 5G is applied to define our Zeeman sublevels, this low field
will not allow state selective addressing by frequency, however allows
for polarization selective addressing. *** Check if 729 will actually
have linewidth for frequency addressing? *** The isotope having 0
nuclear spin and hence no hyperfine levels greatly simplifies control
schemes however precludes the option of using magnetically insensitive
``clock'' qubits. To ensure we do not greatly limit coherence time of
our quadrupole transition we use a MuMetal enclosure to suppress stray
environmental magnetic fields.


%}}}

%Results and Discussion {{{
Results and Discussion
    *Carrier nulling on Blade
%}}}

%Outlook {{{
Outlook
    Current state of building up apparatus
    Proposed first experiments?

%}}}
  
\end{document}
